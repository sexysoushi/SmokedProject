\chapter{Améliorations possibles}

% TODO Lunpe
% Ajout de trajectoires différentes, gestion de plusieurs points d'émission,
% intégration à un environnement, flexibilité

Bien qu'on puisse considérer le projet comme fini, il y a toujours des choses
que nous pourrions ajouter ou améliorer. Ce sont des améliorations qui sont
dans le domaine du nice-to-have et que nous n'avons pas implémenté.

\section{Ajout de trajectoires différentes}
Nous avons une base de shaders très simples qui permettent de déplacer nos
particules correctement. Cela dit, ces déplacements ne sont pas toujours très
impressionnants. Une amélioration possible serait d'ajouter plein de types de
forces différentes et de sélectionner lesquelles appliquer à nos particules
(gravité, force radiale, rappel vers le point d'émission, ajout d'un spin à la
trajectoire ou autres). Certaines de ces choses seraient faciles à implémenter (gravité
ou rappel vers le point d'émission) alors que d'autres demanderaient plus de réflexion
afin de déterminer les bonne forces (et dans quelles directions) appliquer uniquement
en fonction du temps. Il serait ainsi possible de conjuguer toutes ces forces
afin d'avoir des comportements plus poussés.

\section{Gestion de plusieurs points d'émission}
Actuellement, la scène ne comporte qu'un noeud de particules. Augmenter ce
nombre pourrait être intéressant autant d'un point de vue visuel qu'utilitaire.
En effet, monter un système de particules possédant plusieurs sources ou types
de particules permet de faire des choses bien plus aguichantes qu'un simple émetteur
de particules. La raison pour laquelle nous n'avons pas implémenté cette fonction est
que, derrière ses airs triviaux, elle nous pose un problème dans la gestion des shaders.
En effet, créer un noeud deux fois recquiert d'utiliser le shader deux fois, et en faisant
cela on crée le même shader plusieurs fois ce qui n'est pas possible dans notre cas.

\section{Intégration à un environnement}
La dernière amélioration notoire que nous avons pensé faire est d'intégrer nos systèmes
de particules dans un environnement comportant d'autre objets 3D. On pourrait donner
en exemple le fait d'avoir un modèle de torche et de placer un émetteur de particules
ayant des textures se rapprochant de la fumée afin d'avoir l'impression que la torche fume.
Cependant, il serait aussi possible de faire une intégration à un environnement bien plus poussée.
En effet, en passant des objets 3D simples au vertex shader, il serait possible de faire
intéragir nos particules et ces objets: des balles qui rebondissent au sol par exemple. 
